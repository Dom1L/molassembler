\documentclass[a4paper]{article}
\usepackage[a4paper,margin=2cm]{geometry}
\usepackage{mathtools}
\usepackage{breqn}
\usepackage[utf8]{inputenc}

\usepackage{tikz} % only for circled numbers
\newcommand*\circled[1]{
  \tikz[baseline=(char.base)]{
    \node[shape=circle,draw,inner sep=2pt] (char) {#1};
  }
}

\renewcommand{\baselinestretch}{1.5}
\setlength{\parindent}{0pt}

\begin{document}
\small

\newcommand{\posVecDiff}[2]
{\left(\vec{r}_{#1}-\vec{r}_{#2}\right)}

\newcommand{\avgVecDiff}[2]
{\left(\vec{s}_{#1}-\vec{s}_{#2}\right)}

\newcommand{\vecDiff}[2]
{\left(\vec{#1}-\vec{#2}\,\right)}

\newcommand{\vecCross}[2]
{\left(\vec{#1}\times\vec{#2}\,\right)}

\newcommand{\posDependence}
{\left(\left\{\vec{r}_i \right\}\right)}

\newcommand{\distanceErrorFirstTermPart}
{\frac{\posVecDiff{j}{i}^{2}}{U_{ij}^{2}}-1}

\newcommand{\distanceErrorFirstTerm}[1]
{\textrm{max}^{#1}\left(0, \distanceErrorFirstTermPart\right)}

\newcommand{\distanceErrorSecondTermPart}
{\frac{2 L_{ij}^{2}}{L_{ij}^{2} + \posVecDiff{j}{i}^{2}}-1}

\newcommand{\distanceErrorSecondTerm}[1]
{\textrm{max}^{#1} \left(0, \distanceErrorSecondTermPart\right)}

\newcommand{\chiralErrorFirstTerm}[1]
{\textrm{max}^{#1} \left(0, 
    V_{\alpha\beta\gamma\delta}\posDependence{}-U_{V}
  \right)
}

\newcommand{\chiralErrorSecondTerm}[1]
{\textrm{max}^{#1} \left(0,
    L_{V}-V_{\alpha\beta\gamma\delta}\posDependence{} 
  \right)
}

\newcommand{\iNSum}{\sum_{i = 1}^{N}}
\newcommand{\ijNSum}{\sum_{i < j}^{N}}
\newcommand{\chiralSum}
{\sum_{\left(S_\alpha, S_\beta, S_\gamma, S_\delta,
  U_{V}, L_{V}\right) \in{} C} 
}
\newcommand{\dihedralSum}
{\sum_{\left(S_\alpha, S_\beta, S_\gamma, S_\delta,
  U_\phi, L_\phi\right) \in{} D}
}
\newcommand{\phiSymbol}
{\phi_{\alpha\beta\gamma\delta}\posDependence{}}

\newcommand{\dihedralTerm}[1]
{\textrm{max}^{#1} \left(0,
    \left\lvert\phiSymbol{} + \left\{
        \begin{array}{r | r}
        2\pi & \phi < \frac{U_\phi + L_\phi - 2\pi}{2}\\
        0 & \\
        -2\pi & \phi > \frac{U_\phi + L_\phi + 2\pi}{2}\\
        \end{array}
      \right\}-\frac{U_\phi{}+L_\phi}{2}
    \right\rvert-\frac{U_\phi{}-L_\phi}{2}
  \right)
}

\newcommand{\volumeCalculationSets}[4]
{\avgVecDiff{#1}{#4}^{T}
  \cdot \left[
    \avgVecDiff{#2}{#4}
    \times\avgVecDiff{#3}{#4}
  \right]
}

\newcommand{\volumeCalculationSub}[4]
{\posVecDiff{#1}{#4}^{T}
  \cdot \left[
    \posVecDiff{#2}{#4}
    \times\posVecDiff{#3}{#4}
  \right]
}

\newcommand{\volumeCalculation}[4]
{\left(\vec{#1}-\vec{#4} \right)^{T}
  \cdot \left[
    \left(\vec{#2}-\vec{#4} \right)
    \times\left(\vec{#3}-\vec{#4} \right)
  \right]
}

\newcommand{\partialPosVec}[1]
{\frac{\partial}{\partial\vec{r}_{#1}}}

\newcommand{\partialVec}[1]
{\frac{\partial}{\partial\vec{#1}\,}}

\newcommand{\errf}{\textrm{errf}\posDependence}

\section{Cyclic polygon expansion plane}

The plane that the cyclic polygon expands into can be found as follows:

We define the dihedral atom sequence $\vec{A}, \vec{B}, \vec{C}, \vec{D}$ with 
the angles $\alpha = \angle ABC, \beta = \angle BCD$ and the dihedral 
$\varphi = \angle ABCD$ with the following conditions:
\begin{equation*}
  \left|AB\right| = a, \left|BC\right| = b, \left|CD\right| = c
\end{equation*}
\begin{equation*}
\alpha,\beta \in \left[0, \pi\right], \varphi \in \left[-\pi, \pi\right]
\end{equation*}

We arbitrarily choose the following positions for the points:
\begin{align}
  \vec{A} &= \mathbf{R}_{z}(\alpha) a \vec{e}_x\\
  \vec{B} &= \vec{0}\\
  \vec{C} &= b \vec{e}_x\\
  \vec{D} &= \mathbf{R}_{x}(\varphi)\left[\vec{C} + \mathbf{R}_z(\pi - \beta)c\vec{e}_x\right]
\end{align}

where $\vec{e}_x$ is the unit vector along the x axis and $\mathbf{R}_i$ is the
rotation matrix along the axis $i$.

The dihedral distance is formed by the line segment $\overline{AD}$.

In order to find the shortest distance between the line segments $\overline{AD}$
and $\overline{BC}$, we define:
\begin{align}
  \overline{BC}: \vec{r}(\lambda) 
  &= \vec{B} + \lambda\left(\vec{C}-\vec{B}\right)
  = \lambda\vec{C},
  \lambda \in \left[0, 1\right]\\
  \overline{AD}: \vec{s}(\mu) 
  &= \vec{A} + \mu\left(\vec{D}-\vec{A}\right),
  \mu \in \left[0, 1\right]
\end{align}
The shortest distance between the line segments must be orthogonal to both line
segments' direction vectors:
\begin{align}
  \left[\lambda\vec{C} - \left(\vec{A} + \mu(\vec{D}-\vec{A})\right)\right]
  &\cdot\vec{C} = 0\\
  \left[\lambda\vec{C} - \left(\vec{A} +
  \mu(\vec{D}-\vec{A})\right)\right]
  &\cdot\left(\vec{D}-\vec{A}\right) = 0
\end{align}
Solving this system of equations yields:
\begin{align}
  \mu_0 &= - \frac{A_y S_y + A_z S_z}{S_y^2 + S_z^2}\\
  \lambda_0 &= \frac{A_x + \mu_0 S_x}{b}\\
  \textrm{with}\ \vec{S} &= \vec{D} - \vec{A}
\end{align}
These do not satisfy their conditions yet:
\begin{align}
  \lambda_m &= \min\left(\max(\lambda_0, 0), 1\right)\\
  \mu_m &= \left\{
  \begin{array}{r | r}
    -\frac{A_x}{S_x} & \lambda_0 \leq 0\\
    \frac{b - A_x}{S_x} & \lambda_0 \geq 1\\
    \lambda_0 & \textrm{else}
  \end{array}\right\}
\end{align}
The plane in which the cyclic polygon expands can then be defined via the three
points $\vec{A}, \vec{D}$, and $\vec{r}(\lambda_m)$.

For $\varphi = \pm \pi$, these points become collinear and the plane definition
is invalid. Under these circumstances, the plane can instead be defined using
the point $\vec{A}$ and the two vectors $\left(\vec{D}-\vec{A}\right)$ and $\vec{e}_z$.

\end{document}
