\documentclass[a4paper]{article}
\usepackage[a4paper,margin=2cm]{geometry}
\usepackage{mathtools}
\usepackage{breqn}
\usepackage[utf8]{inputenc}

\usepackage{tikz} % only for circled numbers
\newcommand*\circled[1]{
  \tikz[baseline=(char.base)]{
    \node[shape=circle,draw,inner sep=2pt] (char) {#1};
  }
}

\renewcommand{\baselinestretch}{1.5}
\setlength{\parindent}{0pt}


\begin{document}
\small

\newcommand{\posVecDiff}[2]{
  \left( \vec{r}_{#1} - \vec{r}_{#2} \right)
}

\newcommand{\vecDiff}[2]{
  \left( \vec{#1} - \vec{#2}\,\right)
}

\newcommand{\vecCross}[2]{
  \left( \vec{#1} \times \vec{#2}\,\right)
}

\newcommand{\posDependence}{
  \left( \left\{\vec{r}_i \right\} \right)
}

\newcommand{\distanceErrorFirstTermPart}{
  \frac{
    \posVecDiff{j}{i}^{2}
  }{
    U_{ij}^{2}
  } - 1
}

\section{Spiro-atom angle modeling}

The cross-angle in a spiro-atom is expliticly modeled if the disjoint cycles are
both small, i.e. of size <= 5. Only then is the additional variance when
calculating cross-angle bounds particularly harmful for the overall 3D
structure.

We define:

\begin{align}
  \vec{a} &= R_z\left(\frac{\alpha}{2}\right)\begin{pmatrix}
    1\\
    0\\
    0
  \end{pmatrix} = \begin{pmatrix}
    \cos \frac{\alpha}{2}\\
    \sin \frac{\alpha}{2}\\
    0
  \end{pmatrix}\\
  \vec{b} &= R_y\left(\frac{\beta}{2}\right)\begin{pmatrix}
    -1\\
    0\\
    0
  \end{pmatrix} = \begin{pmatrix}
    - \cos \frac{\beta}{2}\\
    0\\
    \sin \frac{\beta}{2}
  \end{pmatrix}
\end{align}

Where $R_y$ and $R_z$ are the standard three-dimensional rotation matrices,
$\alpha$ is the cycle internal angle for the first cycle at the spiro-atom, and
$\beta$ is the cycle internal angle for the second cycle at the spiro-atom.\\

The angle between these vectors $\phi$ is then:

\begin{align}
  \vec{a}\vec{b} &= ||\vec{a}||||\vec{b}|| \cos \phi\\
  \phi &= \arccos \left\{ 
    \left(\cos \frac{\alpha}{2} \right)
    \left(- \cos \frac{\beta}{2} \right)
  \right\}
\end{align}

\end{document}
