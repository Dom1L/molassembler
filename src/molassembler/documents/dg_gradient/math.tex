\documentclass[a4paper]{article}
\usepackage[a4paper,margin=2cm]{geometry}
\usepackage{mathtools}
\usepackage{breqn}
\usepackage[utf8]{inputenc}

\usepackage{tikz} % only for circled numbers
\newcommand*\circled[1]{
  \tikz[baseline=(char.base)]{
    \node[shape=circle,draw,inner sep=2pt] (char) {#1};
  }
}

\renewcommand{\baselinestretch}{1.5}
\setlength{\parindent}{0pt}


\begin{document}
\small

\newcommand{\posVecDiff}[2]{
  \left(\vec{r}_{#1} - \vec{r}_{#2} \right)
}

\newcommand{\avgVecDiff}[2]{
  \left(\vec{s}_{#1} - \vec{s}_{#2} \right)
}

\newcommand{\vecDiff}[2]{
  \left(\vec{#1} - \vec{#2}\,\right)
}

\newcommand{\vecCross}[2]{
  \left(\vec{#1} \times \vec{#2}\,\right)
}

\newcommand{\posDependence}{
  \left(\left\{\vec{r}_i \right\} \right)
}

\newcommand{\distanceErrorFirstTermPart}{
  \frac{
    \posVecDiff{j}{i}^{2}
  }{
    U_{ij}^{2}
  } - 1
}

\newcommand{\distanceErrorFirstTerm}[1]{
  \textrm{max}^{#1} \left(
    0, 
    \distanceErrorFirstTermPart
  \right)
}

\newcommand{\distanceErrorSecondTermPart}{
  \frac{
    2 L_{ij}^{2}
  }{
    L_{ij}^{2} + \posVecDiff{j}{i}^{2}
  } - 1
}

\newcommand{\distanceErrorSecondTerm}[1]{
  \textrm{max}^{#1} \left(
    0,
    \distanceErrorSecondTermPart
  \right)
}

\newcommand{\chiralErrorFirstTerm}[1]{
  \textrm{max}^{#1} \left(
    0,
    V_{\alpha\beta\gamma\delta}\posDependence{} - U_{\alpha\beta\gamma\delta}
  \right)
}

\newcommand{\chiralErrorSecondTerm}[1]{
  \textrm{max}^{#1} \left(
    0,
    L_{\alpha\beta\gamma\delta} - V_{\alpha\beta\gamma\delta}\posDependence{} 
  \right)
}

\newcommand{\iNSum}{\sum_{i = 1}^{N}}
\newcommand{\ijNSum}{\sum_{i < j}^{N}}
\newcommand{\chiralSum}{
  \sum_{\left(S_\alpha, S_\beta, S_\gamma, S_\delta,
  U_{\alpha\beta\gamma\delta}, L_{\alpha\beta\gamma\delta}\right) \in C} 
}

\newcommand{\volumeCalculationSets}[4]{
  \avgVecDiff{#1}{#4}^{T}
  \cdot \left[
    \avgVecDiff{#2}{#4}
    \times\avgVecDiff{#3}{#4}
  \right]
}

\newcommand{\volumeCalculationSub}[4]{
  \posVecDiff{#1}{#4}^{T}
  \cdot \left[
    \posVecDiff{#2}{#4}
    \times\posVecDiff{#3}{#4}
  \right]
}

% TODO refactor to vecDiff
\newcommand{\volumeCalculation}[4]{
  \left(\vec{#1} - \vec{#4} \right)^{T}
  \cdot \left[
    \left(\vec{#2} - \vec{#4} \right)
    \times\left(\vec{#3} - \vec{#4} \right)
  \right]
}

\newcommand{\partialPosVec}[1]{
  \frac{\partial}{\partial \vec{r}_{#1}}
}

\newcommand{\partialVec}[1]{
  \frac{\partial}{\partial \vec{#1}\,}
}

\newcommand{\errf}{\textrm{errf}\posDependence}

\section{DG Error function}

For a given set of $N$ particles with positions $\vec{r}_i$, the distance
geometry error function is given as:

\begin{align}\begin{aligned} % nested to get only a single numbered equation
  \errf &= \underbrace {
    \ijNSum \left[
      \distanceErrorFirstTerm{2} + \distanceErrorSecondTerm{2}
    \right]
  }_{\textrm{Distance errors}}\\
  &+ \underbrace{
    \chiralSum \left[ 
      \chiralErrorFirstTerm{2} + \chiralErrorSecondTerm{2}
    \right]
  }_{\textrm{Chiral errors}}.
\end{aligned}\end{align}

Within the distance errors, the symbols $U_{ij}$ and $L_{ij}$ are the upper and
lower distance bounds for the atoms $i$ and $j$.\\

Within the chiral errors, $C$ is a set of chiral constraint tuples consisting of
the sets $S_\alpha, S_\beta, S_\gamma$ and $S_\delta$. These contain mutually
disjoint particle indices and contain at least one element. In further notation,
$\vec{s}$ denotes the average spatial position of all elements of a set, e.g.\
for $S_\alpha$:

\begin{equation}
  \vec{s}_\alpha = \frac{1}{|S_\alpha|}\sum_{i=1}^{|S_\alpha|}
  \vec{r}_{S_{\alpha, i}},
\end{equation}

where $|S_\alpha|$ denotes the number of elements in the set and $S_{\alpha,
i}$ is the $i$-th element in the set.\\

The constraint tuple further consists of the scalars $U_{\alpha\beta\gamma\delta}$ and
$L_{\alpha\beta\gamma\delta}$, which are upper and lower bounds on
the volume spanned by the average positions of each set. This volume is
calculated in the symbol $V_{\alpha\beta\gamma\delta}$, which is the signed
tetrahedron volume spanned by $\vec{s}_\alpha, \vec{s}_\beta, \vec{s}_\gamma$
and $\vec{s}_\delta$:

\begin{equation}
  V_{\alpha\beta\gamma\delta}\left( \left\{\vec{r}_i \right\} \right) =
    \volumeCalculationSets{\alpha}{\beta}{\gamma}{\delta}.
\end{equation}

It is important to note that tetrahedron volumes such as
$U_{\alpha\beta\gamma\delta}$ are signed values. On odd permutations of
indices, these quantities change sign.

\section{Gradient of the error function}

$E$ is scalar-valued, so the columnar gradient of the error function is
composed of partial derivatives to the individual position vectors $\vec{r}_i$:

\begin{align*}
  \nabla E &= \left(
    \frac{\partial E}{\partial \vec{r}_1},
    \frac{\partial E}{\partial \vec{r}_2},
    \ldots,
    \frac{\partial E}{\partial \vec{r}_N}
  \right)
\end{align*}

Each individual component $\left( \partial E / \partial \vec{r}_\xi \right)$
is a vector whose components are the scalar derivatives:

\begin{align*}
  \frac{\partial E}{\partial \vec{r}_\xi} &= \begin{pmatrix}
    \partial E / \partial r_{\xi , x} \\
    \partial E / \partial r_{\xi , y} \\
    \partial E / \partial r_{\xi , z} 
  \end{pmatrix}
\end{align*}
\newpage

We split the problem into the main terms:

\begin{align*}
  \partialPosVec{\xi} \errf 
  &= \underbrace{
    \partialPosVec{\xi} \left(
      \ijNSum \distanceErrorFirstTerm{2}
    \right)
  }_{\circled{1}}
  + \underbrace{
    \partialPosVec{\xi} \left(
      \ijNSum \distanceErrorSecondTerm{2}
    \right)
  }_{\circled{2}}\\
  &+ \underbrace{
    \partialPosVec{\xi} \chiralSum \chiralErrorFirstTerm{2}
  }_{\circled{3}}\\
  &+ \underbrace{
    \partialPosVec{\xi} \chiralSum \chiralErrorSecondTerm{2}
  }_{\circled{4}}
\end{align*}

and start with \circled{1}. We use the chain rule:

\begin{equation}
  \partialPosVec{\xi} \ijNSum \distanceErrorFirstTerm{2}
  = 2 \ijNSum \distanceErrorFirstTerm{} \partialPosVec{\xi}
  \distanceErrorFirstTerm{}.
\end{equation}

If $\xi = i$, then:

\begin{equation}
  \partialPosVec{i} \distanceErrorFirstTermPart
  = - \frac{2}{U_{ij}^{2}} \posVecDiff{j}{i},
\end{equation}

and likewise, but positive, for $\xi = j$. Consequently:

\begin{equation}
  \partialPosVec{\xi} \left( \distanceErrorFirstTerm{} \right)
  = \frac{2}{U_{ij}^{2}} \posVecDiff{j}{i} \left\{ \begin{array}{r | r}
    -1 & \xi = i\\
    1 & \xi = j\\
    0 & \textrm{else}
  \end{array} \right\}
  = \frac{2}{U_{ij}^{2}} \posVecDiff{j}{i} \left(
    \delta_{\xi j} - \delta_{\xi i} 
  \right),
\end{equation}

where we have discarded the possibility of $\distanceErrorFirstTermPart < 0$
since this case is adequately covered by the first maximum function. $\delta$ is
the Kronecker delta. So, in total:

\begin{equation}\label{eq_breakdown_deltas}
  \text{\circled{1}}
  = \ijNSum \left(
    \delta_{\xi j} - \delta_{\xi i} 
  \right) \underbrace{
    \frac{4}{U_{ij}^{2}} \posVecDiff{j}{i} \distanceErrorFirstTerm{}
  }_{f(i, j)}
\end{equation}

We can transform the summation further using the Kronecker deltas:
\begin{align}
  \ijNSum \left( \delta_{\xi j} - \delta_{\xi i} \right) f(i, j)
  &= \sum_{i = 1}^{\xi - 1} f(i, \xi) 
    - \sum_{j = \xi + 1}^{N} f(\xi, j)\\
  &= \sum_{i = 1}^{\xi - 1} f(i, \xi) 
    + \sum_{i = \xi + 1}^{N} f(i, \xi)\\
  &= \sum_{i = 1}^{N} \left(1 - \delta_{i\xi} \right) f(i, \xi),
\end{align}

where we have used that $f(i, j) = -f(j, i)$ (see Eq.~\ref{eq_breakdown_deltas}).
All in all:

\begin{equation}
  \text{\circled{1}}
  = \iNSum \left(1 - \delta_{i\xi}\right) \frac{4}{U_{i\xi}^{2}}
    \posVecDiff{\xi}{i} \textrm{max} \left(
      0,
      \frac{
        \posVecDiff{\xi}{i}^{2}
      }{
        U_{i\xi}^{2}
      } - 1
    \right)
\end{equation}

\newpage
Let us continue with \circled{2}. The chain rule gives:
\begin{equation}
  \partialPosVec{\xi} \ijNSum \distanceErrorSecondTerm{2}
  = 2 \ijNSum \distanceErrorSecondTerm{} \partialPosVec{\xi}
    \distanceErrorSecondTerm.
\end{equation}

For $\xi = i$,
\begin{equation}
  \partialPosVec{i} \left( \distanceErrorSecondTermPart \right)
  = \frac{
    4 L_{ij}^{2} \posVecDiff{j}{i}
  }{
    \left(
      L_{ij}^{2} + \posVecDiff{j}{i}^2
    \right)^2
  },
\end{equation}

and likewise, but negative, for $\xi = j$. Therefore,

\begin{align*}
  \partialPosVec{\xi} \left( \distanceErrorSecondTerm{} \right)
  &=\frac{
    4 L_{ij}^{2} \posVecDiff{j}{i}
  }{
    \left(
      L_{ij}^{2} + \posVecDiff{j}{i}^2
    \right)^2
  } \left\{ \begin{array}{ r | r}
    1 & \xi = i\\
    - 1 & \xi = j\\
    0 & \textrm{else}
  \end{array} \right\}\\
  &= \frac{
    4 L_{ij}^{2} \posVecDiff{j}{i}
  }{
    \left(
      L_{ij}^{2} + \posVecDiff{j}{i}^2
    \right)^2
  } \left(
    \delta_{\xi i} - \delta_{\xi j} 
  \right),
\end{align*}

where we have excluded the possibility of $\distanceErrorSecondTermPart < 0$
since this case is covered by the first maximum function. Altogether:

\begin{equation}\label{eq_breakdown_deltas_second}
  \text{\circled{2}}
  = \ijNSum \left(
    \delta_{\xi i} - \delta_{\xi j} 
  \right) \underbrace{
    \frac{
      8 L_{ij}^{2} \posVecDiff{j}{i}
    }{
      \left(
        L_{ij}^{2} + \posVecDiff{j}{i}^2
      \right)^2
    } \distanceErrorSecondTerm{}
  }_{g(i, j)}.
\end{equation}

Transforming the summation:
\begin{align}
  \ijNSum \left( \delta_{\xi i} - \delta_{\xi j} \right) g(i, j)
  &= - \sum_{i = 1}^{\xi - 1} g(i, \xi) 
    + \sum_{j = \xi + 1}^{N} g(\xi, j)\\
  &= \sum_{i = 1}^{\xi - 1} g(\xi, i) 
    + \sum_{i = \xi + 1}^{N} g(\xi, i)\\
  &= \sum_{i = 1}^{N} \left(1 - \delta_{i\xi} \right) g(\xi, i),
\end{align}

Where we have used $g(i, j) = -g(j, i)$ (see
Eq.~\ref{eq_breakdown_deltas_second}). All in all:

\begin{equation}
  \text{\circled{2}}
  = \iNSum \left(1 - \delta_{i\xi}\right) 
    \frac{
      8 L_{\xi i}^{2} \posVecDiff{i}{\xi}
    }{
      \left(
        L_{\xi i}^{2} + \posVecDiff{i}{\xi}^2
      \right)^2
    } \textrm{max} \left(
      0,
      \frac{
        2 L_{\xi i}^{2}
      }{
        L_{\xi i}^{2} + \posVecDiff{i}{\xi}^{2}
      } - 1
    \right)
\end{equation}

\newpage
Next, we consider \circled{3}. Applying the chain rule gives:

\begin{align*}
  &\quad\partialPosVec{\xi} \chiralSum \chiralErrorFirstTerm{2}\\
  &= 2 \chiralSum \chiralErrorFirstTerm{} \partialPosVec{\xi} \chiralErrorFirstTerm{}
\end{align*}

The partial derivatives of $V_{\alpha\beta\gamma\delta}\posDependence$ with
respect to $\vec{r}_\xi$ are split into five cases. The index $\xi$ can be an
element of one of the four sets $S_\alpha, S_\beta, S_\gamma, S_\delta$ (and
only one, since they are mutually disjoint) or not. The derivative for the last
case is zero. In the remaining cases, one average vector $\vec{s}$ is a function
of $\vec{r}_\xi$ but the rest are not. The partial derivative of any average
vector $\vec{s}$ is:

\begin{equation}
  \partialPosVec{\xi} \vec{s}_\alpha 
  = \partialPosVec{\xi} \frac{1}{|S_\alpha|}\sum_{i=1}^{|S_\alpha|}
  \vec{r}_{S_{\alpha, i}}
  = \left\{ 
    \begin{array}{r | r}
      \frac{1}{|S_\alpha|} \mathbf{I}_3 &  \xi \in S_\alpha\\
      0 & \textrm{else}
    \end{array}
  \right\}
\end{equation}

where $\mathbf{I}_3$ denotes the three dimensional identity matrix. The individual set
membership cases are thus as follows:

\begin{align}
  \text{\circled{$S_\alpha$}} &\quad \partialPosVec{\xi} \left\{ 
    \avgVecDiff{\alpha}{\delta}^{T}
    \cdot \left[
      \avgVecDiff{\beta}{\delta}
      \times \avgVecDiff{\gamma}{\delta}
    \right]
  \right\} \\
  &= \partialPosVec{\xi} \left\{ 
    \vec{s}_\alpha^{\,T}
    \cdot \left[
      \avgVecDiff{\beta}{\delta}
      \times \avgVecDiff{\gamma}{\delta}
    \right]
  \right\} - \vec{0} \\
  &= \frac{1}{|S_\alpha|} \left[
    \avgVecDiff{\beta}{\delta} \times \avgVecDiff{\gamma}{\delta} 
  \right]
\end{align}

In shorthand notation, in which all $\vec{s}$ symbols are replaced by their subscripts:

\begin{align}
  \text{\circled{$S_\beta$}} &\quad \partialPosVec{\xi} 
    \vecDiff{\alpha}{\delta}^{T}
    \cdot \left[
      \vecDiff{\beta}{\delta}
      \times \vecDiff{\gamma}{\delta}
    \right] \\
  &= \partialPosVec{\xi}
    \vecDiff{\alpha}{\delta}^{T}
    \cdot \left[
      \vec{\beta} \times \vec{\gamma} - \vec{\beta} \times \vec{\delta} - \vec{\delta} \times \vec{\gamma} + \underbrace{
        \vec{\delta} \times \vec{\delta}
      }_{=0}
    \right] \\
  &= \partialPosVec{\xi} \left\{
    \vec{\beta}^{\,T} \cdot \left[
      \vec{\gamma} \times \vecDiff{\alpha}{\delta}
    \right] 
    - \vec{\beta}^{\,T} \cdot \left[
      \vec{\delta} \times \vecDiff{\alpha}{\delta}
    \right] 
  \right\} - \vec{0} \\
&= \frac{1}{|S_\beta|} \vecDiff{\gamma}{\delta} \times \vecDiff{\alpha}{\delta} 
  = - \frac{1}{|S_\beta|}\vecDiff{\alpha}{\delta} \times \vecDiff{\gamma}{\delta}
\end{align}

\begin{align}
  \text{\circled{$S_\gamma$}} &\quad \partialPosVec{\xi} 
    \vecDiff{\alpha}{\delta}^{T}
    \cdot \left[
      \vec{\beta} \times \vec{\gamma} - \vec{\beta} \times \vec{\delta} - \vec{\delta} \times \vec{\gamma} 
    \right] \\
  &= \partialPosVec{\xi} \left\{
    \vec{\gamma}^{\,T} \cdot \left[
      \vecDiff{\alpha}{\delta} \times \vec{\beta}
    \right] 
    - \vec{\gamma}^{\,T} \cdot \left[
      \vecDiff{\alpha}{\delta} \times \vec{\delta}\,
    \right] 
  \right\} - \vec{0} \\
  &= \vecDiff{\alpha}{\delta} \times \vecDiff{\beta}{\delta} 
\end{align}

\begin{align}
  \text{\circled{$S_\delta$}} &\quad \partialPosVec{\xi}
    \vecDiff{\alpha}{\delta}^{T}
    \cdot \left[
      \vec{\beta} \times \vec{\gamma} - \vec{\beta} \times \vec{\delta} - \vec{\delta} \times \vec{\gamma} 
    \right] \\
  &= \partialPosVec{\xi} \vec{\alpha}^{\,T} \cdot \left[
      \vec{\beta} \times \vec{\gamma} - \vec{\beta} \times \vec{\delta} - \vec{\delta} \times \vec{\gamma} 
    \right] - \partialPosVec{\xi} \vec{\delta}^{\,\,T} \cdot \left[
      \vec{\beta} \times \vec{\gamma} - \vec{\beta} \times \vec{\delta} - \vec{\delta} \times \vec{\gamma}
    \right]\\
  &= \underbrace{
      \partialPosVec{\xi} \vec{\alpha}^{\,T} \cdot \vecCross{\beta}{\gamma}
    }_{=0} 
    - \partialPosVec{\xi} \vec{\alpha}^{\,T} \cdot \left( 
      \vec{\beta} \times \vec{\delta}\,
    \right)
    - \partialPosVec{\xi} \vec{\alpha}^{\,T} \cdot \vecCross{\delta}{\gamma}
    - \partialPosVec{\xi} \vec{\delta}^{\,\,T} \cdot \vecCross{\beta}{\gamma}
    + \partialPosVec{\xi} \underbrace{
      \vec{\delta}^{\,\,T} \cdot \vecCross{\delta}{\gamma} 
    }_{=0}
    + \partialPosVec{\xi} \underbrace{
      \vec{\delta}^{\,\,T} \cdot \vecCross{\beta}{\delta}
    }_{=0} \\
  &= - \partialPosVec{\xi} \vec{\delta}^{\,\,T} \cdot \vecCross{\gamma}{\alpha}
    - \partialPosVec{\xi} \vec{\delta}^{\,\,T} \cdot \vecCross{\alpha}{\beta}
    - \partialPosVec{\xi} \vec{\delta}^{\,\,T} \cdot \vecCross{\beta}{\gamma} \\
  &= - \frac{1}{|S_\delta|} \vec{\gamma} \times \vec{\alpha} -
  \frac{1}{|S_\delta|} \vec{\alpha} \times \vec{\beta} - \frac{1}{|S_\delta|} \vec{\beta} \times \vec{\gamma} \\
  &= - \frac{1}{|S_\delta|} \vecDiff{\alpha}{\gamma} \times \vecDiff{\beta}{\gamma} 
\end{align}

So, overall:

\begin{equation}
  \text{\circled{3}} = \chiralSum 2\ \chiralErrorFirstTerm{}  \left\{ \begin{array}{r | r}
    \frac{1}{|S_\alpha|} \avgVecDiff{\beta}{\delta} \times \avgVecDiff{\gamma}{\delta} & \xi \in S_\alpha\\
    - \frac{1}{|S_\beta|} \avgVecDiff{\alpha}{\delta} \times \avgVecDiff{\gamma}{\delta} & \xi \in S_\beta\\
    \frac{1}{|S_\gamma|} \avgVecDiff{\alpha}{\delta} \times \avgVecDiff{\beta}{\delta} & \xi \in S_\gamma\\
    - \frac{1}{|S_\delta|} \avgVecDiff{\alpha}{\gamma} \times \avgVecDiff{\beta}{\gamma} & \xi \in S_\delta\\
    0 & \textrm{else}
  \end{array} \right\}.
\end{equation}

For \circled{4}, the derivation is analog save for the sign of the great amount
of cases, which we extrude from the sum:

\begin{equation}
  \text{\circled{4}} = - \chiralSum 2\ \chiralErrorSecondTerm{} \left\{ \begin{array}{r | r}
    \frac{1}{|S_\alpha|} \avgVecDiff{\beta}{\delta} \times \avgVecDiff{\gamma}{\delta} & \xi \in S_\alpha\\
    - \frac{1}{|S_\beta|} \avgVecDiff{\alpha}{\delta} \times \avgVecDiff{\gamma}{\delta} & \xi \in S_\beta\\
    \frac{1}{|S_\gamma|} \avgVecDiff{\alpha}{\delta} \times \avgVecDiff{\beta}{\delta} & \xi \in S_\gamma\\
    - \frac{1}{|S_\delta|} \avgVecDiff{\alpha}{\gamma} \times \avgVecDiff{\beta}{\gamma} & \xi \in S_\delta\\
    0 & \textrm{else}
  \end{array} \right\}.
\end{equation}

Both terms concerning the chiral error can be summarized as follows:

\begin{align*}
  &\quad\chiralSum 2\ \chiralErrorFirstTerm{} \Big\{ \ldots \Big\}\\
  &\quad - \chiralSum 2\ \chiralErrorSecondTerm{} \Big\{ \ldots \Big\}\\
  &= \chiralSum 2\ \Big\{ \ldots \Big\} \Big[ 
    \chiralErrorFirstTerm{} - \chiralErrorSecondTerm{} 
  \Big],
\end{align*}

which ought to save some time in an implementation.

\end{document}
