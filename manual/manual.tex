\documentclass[]{tufte-book}

\hypersetup{colorlinks}% uncomment this line if you prefer colored hyperlinks (e.g., for onscreen viewing)

%%
% For graphics / images
\usepackage{graphicx}
\setkeys{Gin}{width=\linewidth,totalheight=\textheight,keepaspectratio}
\graphicspath{{graphics/}}
\usepackage{hyperref}
\usepackage{chemformula}

%%
% Book metadata
\title[SCINE Molassembler manual]{User Manual \vskip 0.5em
{\setlength{\parindent}{0pt} \Huge SCINE Molassembler 1.0.0}}
\author[The SCINE Molassembler Developers]{The SCINE Molassembler Developers:
\newline \noindent Jan-Grimo Sobez and Markus Reiher}
\publisher{ETH Z\"urich}

%%
% If they're installed, use Bergamo and Chantilly from www.fontsite.com.
% They're clones of Bembo and Gill Sans, respectively.
%\IfFileExists{bergamo.sty}{\usepackage[osf]{bergamo}}{}% Bembo
%\IfFileExists{chantill.sty}{\usepackage{chantill}}{}% Gill Sans

%\usepackage{microtype}

%%
% Just some sample text
\usepackage{lipsum}

%%
% For nicely typeset tabular material
\usepackage{booktabs}

% The fancyvrb package lets us customize the formatting of verbatim
% environments.  We use a slightly smaller font.
\usepackage{fancyvrb}
\fvset{fontsize=\normalsize}
\fvset{frame=leftline}
\fvset{framesep=2mm}
\fvset{xleftmargin=-2mm}

%%
% Prints argument within hanging parentheses (i.e., parentheses that take
% up no horizontal space).  Useful in tabular environments.
\newcommand{\hangp}[1]{\makebox[0pt][r]{(}#1\makebox[0pt][l]{)}}

%%
% Prints an asterisk that takes up no horizontal space.
% Useful in tabular environments.
\newcommand{\hangstar}{\makebox[0pt][l]{*}}

%%
% Prints a trailing space in a smart way.
\usepackage{xspace}

% Prints the month name (e.g., January) and the year (e.g., 2008)
\newcommand{\monthyear}{%
  \ifcase\month\or January\or February\or March\or April\or May\or June\or
  July\or August\or September\or October\or November\or
  December\fi\space\number\year
}


% Prints an epigraph and speaker in sans serif, all-caps type.
\newcommand{\openepigraph}[2]{%
  %\sffamily\fontsize{14}{16}\selectfont
  \begin{fullwidth}
  \sffamily\large
  \begin{doublespace}
  \noindent\allcaps{#1}\\% epigraph
  \noindent\allcaps{#2}% author
  \end{doublespace}
  \end{fullwidth}
}

% Inserts a blank page
\newcommand{\blankpage}{\newpage\hbox{}\thispagestyle{empty}\newpage}

\usepackage{units}

% Typesets the font size, leading, and measure in the form of 10/12x26 pc.
\newcommand{\measure}[3]{#1/#2$\times$\unit[#3]{pc}}

% Macros for typesetting the documentation
\newcommand{\hlred}[1]{\textcolor{Maroon}{#1}}% prints in red
\newcommand{\hangleft}[1]{\makebox[0pt][r]{#1}}
\newcommand{\hairsp}{\hspace{1pt}}% hair space
\newcommand{\hquad}{\hskip0.5em\relax}% half quad space
\newcommand{\TODO}{\textcolor{red}{\bf TODO!}\xspace}
\newcommand{\ie}{\textit{i.\hairsp{}e.}\xspace}
\newcommand{\eg}{\textit{e.\hairsp{}g.}\xspace}
\newcommand{\na}{\quad--}% used in tables for N/A cells
\providecommand{\XeLaTeX}{X\lower.5ex\hbox{\kern-0.15em\reflectbox{E}}\kern-0.1em\LaTeX}
\newcommand{\tXeLaTeX}{\XeLaTeX\index{XeLaTeX@\protect\XeLaTeX}}
% \index{\texttt{\textbackslash xyz}@\hangleft{\texttt{\textbackslash}}\texttt{xyz}}
\newcommand{\tuftebs}{\symbol{'134}}% a backslash in tt type in OT1/T1
\newcommand{\doccmdnoindex}[2][]{\texttt{\tuftebs#2}}% command name -- adds backslash automatically (and doesn't add cmd to the index)
\newcommand{\doccmddef}[2][]{%
  \hlred{\texttt{\tuftebs#2}}\label{cmd:#2}%
  \ifthenelse{\isempty{#1}}%
    {% add the command to the index
      \index{#2 command@\protect\hangleft{\texttt{\tuftebs}}\texttt{#2}}% command name
    }%
    {% add the command and package to the index
      \index{#2 command@\protect\hangleft{\texttt{\tuftebs}}\texttt{#2} (\texttt{#1} package)}% command name
      \index{#1 package@\texttt{#1} package}\index{packages!#1@\texttt{#1}}% package name
    }%
}% command name -- adds backslash automatically
\newcommand{\doccmd}[2][]{%
  \texttt{\tuftebs#2}%
  \ifthenelse{\isempty{#1}}%
    {% add the command to the index
      \index{#2 command@\protect\hangleft{\texttt{\tuftebs}}\texttt{#2}}% command name
    }%
    {% add the command and package to the index
      \index{#2 command@\protect\hangleft{\texttt{\tuftebs}}\texttt{#2} (\texttt{#1} package)}% command name
      \index{#1 package@\texttt{#1} package}\index{packages!#1@\texttt{#1}}% package name
    }%
}% command name -- adds backslash automatically
\newcommand{\docopt}[1]{\ensuremath{\langle}\textrm{\textit{#1}}\ensuremath{\rangle}}% optional command argument
\newcommand{\docarg}[1]{\textrm{\textit{#1}}}% (required) command argument
\newenvironment{docspec}{\begin{quotation}\ttfamily\parskip0pt\parindent0pt\ignorespaces}{\end{quotation}}% command specification environment
\newcommand{\docenv}[1]{\texttt{#1}\index{#1 environment@\texttt{#1} environment}\index{environments!#1@\texttt{#1}}}% environment name
\newcommand{\docenvdef}[1]{\hlred{\texttt{#1}}\label{env:#1}\index{#1 environment@\texttt{#1} environment}\index{environments!#1@\texttt{#1}}}% environment name
\newcommand{\docpkg}[1]{\texttt{#1}\index{#1 package@\texttt{#1} package}\index{packages!#1@\texttt{#1}}}% package name
\newcommand{\doccls}[1]{\texttt{#1}}% document class name
\newcommand{\docclsopt}[1]{\texttt{#1}\index{#1 class option@\texttt{#1} class option}\index{class options!#1@\texttt{#1}}}% document class option name
\newcommand{\docclsoptdef}[1]{\hlred{\texttt{#1}}\label{clsopt:#1}\index{#1 class option@\texttt{#1} class option}\index{class options!#1@\texttt{#1}}}% document class option name defined
\newcommand{\docmsg}[2]{\bigskip\begin{fullwidth}\noindent\ttfamily#1\end{fullwidth}\medskip\par\noindent#2}
\newcommand{\docfilehook}[2]{\texttt{#1}\index{file hooks!#2}\index{#1@\texttt{#1}}}
\newcommand{\doccounter}[1]{\texttt{#1}\index{#1 counter@\texttt{#1} counter}}

%attempt to allow footnotes in verbatim
\usepackage{verbatim}
\newcommand{\vfchar}[1]{
  % the usual trick for using a "variable" active character
  \begingroup\lccode`~=`#1 \lowercase{\endgroup\def~##1~}{%
    % separate the footnote mark from the footnote text
    % so the footnote mark will occupy the same space as
    % any other character
    \makebox[0.5em][l]{\footnotemark}%
    \footnotetext{##1}%
  }%
  \catcode`#1=\active
}
\newenvironment{fverbatim}[1]
 {\verbatim\vfchar{#1}}
 {\endverbatim}


% Generates the index
\usepackage{makeidx}
\makeindex

%\usepackage{natbib}
\setcitestyle{numbers,square}

\usepackage{parskip}



\begin{document}

\setlength{\parindent}{0pt}

% Front matter
\frontmatter


% r.3 full title page
\maketitle


% v.4 copyright page
\newpage
\begin{fullwidth}
~\vfill
\thispagestyle{empty}
\setlength{\parindent}{0pt}
\setlength{\parskip}{\baselineskip}
Copyright \copyright\ \the\year\ \thanklessauthor

%\par\smallcaps{Published by \thanklesspublisher}

\par\smallcaps{https://scine.ethz.ch/download/molassembler}

\par Unless required by applicable law or agreed to in writing, the software 
is distributed on an \smallcaps{``AS IS'' BASIS, WITHOUT
WARRANTIES OR CONDITIONS OF ANY KIND}, either express or implied. \index{license}

%\par\textit{First printing, \monthyear}
\end{fullwidth}

% r.5 contents
\tableofcontents

%\listoffigures

%\listoftables


%%
% Start the main matter (normal chapters)
\mainmatter

\let\cleardoublepage\clearpage
\chapter{Introduction}

\textsc{Molassembler} is a software package that aims to facilitate conversions
between Cartesian and graph representations of molecules. It provides the
necessary functionality to represent a molecule as a graph, modify it in graph
space, and generate new coordinates from graphs. It can capture the absolute
configuration of multidentate and haptic inorganic molecules from positional
data and generate non-superposable stereopermutations as output.

SCINE \textsc{Molassembler} consists of a C++ library and a Python module which
binds parts of the available C++ functionality. 

In this manual, we describe the installation of the software and how to
integrate it into any derivative projects. A prospect on
features in future releases and references for further reading are added at the
end of this manual.\enlargethispage{\baselineskip}



\chapter{Obtaining the software}\label{ch:obtain}

\textsc{Molassembler}  is distributed as open source software in the framework
of the \href{https://scine.ethz.ch/}{SCINE
Project}\footnote{\texttt{scine.ethz.ch}}. It is licensed under the BSD-3 clause
license. Visit our
\href{https://scine.ethz.ch/download/molassembler}{website}\footnote{\texttt{scine.ethz.ch/download/molassembler}}
to obtain the software. 


\section{System requirements}

\textsc{Molassembler} has only modest requirements regarding the hardware
performance, and will chew through small problems on a single thread without
much trouble. However, if the studied systems are large, it is worth enabling
parallelization in order to reduce wall clock times.


\chapter{Installation}\label{ch:installation}

\section{Python bindings on PyPI}

If you want to experiment with just the Python bindings, those are available as
a package from PyPI for Linux platforms.\footnote{Caveat: Your CPU architecture
must be x86--64 or i686.} If you are using Linux, then you can install
\textsc{Molassembler} with:

\begin{Verbatim}
python3 -m pip upgrade --user pip
python3 -m pip install --user scine_molassembler
\end{Verbatim}

When installed this way, the python bindings are not optimized to your
particular CPU instruction set and may therefore be slower than if you had
compiled them yourself.

\section{Conan package}
Conan\footnote{\texttt{conan.io}} is a Python-based package distribution
solution. It integrates well with C++ code built with CMake, greatly simplifying
the installation of libraries with multiple dependencies. Conan is installed
with \texttt{pip}:

\begin{Verbatim}
python3 -m pip install conan
\end{Verbatim}

\textsc{Molassembler} packages are distributed from our research group's remote.
You can add this remote with:

\begin{Verbatim}
conan remote add scine https://scine-artifactory.ethz.ch/artifactory/api/conan/public
\end{Verbatim}

Before we install \textsc{Molassembler}, consider the options the package 
provides. They are summarized below, with valid values and the default option
value in bold face:

\begin{itemize} 
  \item \texttt{shared=[\textbf{True}, False]}: Build molassembler as a shared
    library
  \item \texttt{python=[True, \textbf{False}]}: Build the python bindings
  \item \texttt{docs=[True, \textbf{False}]}: Build documentation
  \item \texttt{tests=[True, \textbf{False}]}: Build and run tests
  \item \texttt{microarch=[detect, \textbf{none}]}: Tune build to CPU
    instruction set
\end{itemize}

The base command to install \textsc{Molassembler} is:

\begin{Verbatim}
conan install scine_molassembler/1.0.0@scine/stable
\end{Verbatim}

Changes to default options can be supplied right after \texttt{conan install}
with \texttt{-o scine\_molassembler:<option>=<value>}.

For instance, if you want python bindings, your install command will be:

\begin{Verbatim}
conan install -o scine_molassembler:python=True scine_molassembler/1.0.0@scine/stable
\end{Verbatim}

Conan will proceed to identify your operating system, architecture and compiler.
It will scan \textsc{Molassembler}'s dependencies and try to find a prebuilt
package for your system. If any package is not available in binary form for your
system, \texttt{conan} can handle its build with the addition of
\texttt{--build=missing} as advised by Conan's error message. If the package is
available in binary form, it will be downloaded.


\section{From source code with CMake}
In order to compile \textsc{Molassembler} from source, you will need:
\begin{itemize}
  \item a C++ compiler supporting the C++14 standard\footnote{\texttt{gcc} $\ge$
    7.3.0, \texttt{clang}~$\ge$~4, \texttt{MinGW-w64}, etc.}
  \item \texttt{cmake} $\ge$ 3.9.0
  \item the \texttt{Boost} libraries $\ge$ 1.65.0
  \item the \texttt{Eigen} library $\ge$ 3.3.2
\end{itemize}

Check your package manager for the relevant packages. For Ubuntu, for instance,
you will need:

\begin{Verbatim}
build-essential
cmake
libboost-all-dev
libeigen3-dev
\end{Verbatim}

First, we must get a complete set of source files. You can get a complete set of
files with:
\begin{Verbatim}
git clone -{}-recursive https://github.com/qcscine/molassembler.git
\end{Verbatim} 
or by extracting a downloaded
release archive and executing 
\begin{Verbatim}
git clone https://github.com/qcscine/development-utils.git
\end{Verbatim} within the
\texttt{molassembler} folder.

Next, we'll get to configuring the source build:

\begin{Verbatim}
mkdir build install
cd build
cmake -DCMAKE_BUILD_TYPE=Release -DCMAKE_INSTALL_PREFIX=../install ..
\end{Verbatim}

This will look through the dependencies to make sure everything needed is there
and prepare for building. We have specified an installation prefix on the
command line that we will install into, which you can replace at your leisure.

If you additionally wish to compile and install the python bindings, you can
rerun \texttt{cmake} with the following flags added:

\begin{Verbatim}
cmake -DSCINE_BUILD_PYTHON_BINDINGS=ON -DPYTHON_EXECUTABLE=$(which python3) ..
\end{Verbatim}

For further compile-time options regarding the build of \textsc{Molassembler},
you can run \texttt{cmake -L ..} in the \texttt{build} directory. Look for
options prefixed with \texttt{SCINE} or \texttt{MOLASSEMBLER}. You can then
build, test and install with:

\begin{Verbatim}
make
make test
make install
\end{Verbatim}

In case you need support with the setup of \textsc{Molassembler}, please contact
us by writing to \href{mailto:scine@phys.chem.ethz.ch}{scine@phys.chem.ethz.ch}.

Note that \textsc{Molassembler} depends on further libraries that are
distributed along or downloaded at configure time:

\begin{itemize}
  \item RingDecomposerLib~\cite{Flachsenberg2017, Kolodzik2012} 
  \item nauty~\cite{McKay2014}
  \item JSON For Modern C++~\cite{Lohmann2013} 
  \item SCINE Utilities~\cite{Brunken2019}
\end{itemize}


\chapter{Using the C++ library}

\textsc{Molassembler} provides a
\href{https://scine.ethz.ch/static/download/documentation/molassembler/v1.0.0/cpp}{\texttt{Doxygen}
technical
documentation}\footnote{\texttt{scine.ethz.ch/download/molassembler}}
which documents the available functions in detail and also provides an overview
of the available functionality. Here, we will only go into detail in how to link
\textsc{Molassembler} into a new project with \texttt{CMake}.

In the \texttt{CMake} file of a new project, add the following:

\begin{Verbatim}
# Suppose your target is named transmogrifier
find_package(molassembler REQUIRED)
target_link_libraries(transmogrifier PUBLIC molassembler::molassembler)
\end{Verbatim}

Note that if \texttt{transmogrifier} were a library you were making,
\texttt{PUBLIC} here denotes that you use \textsc{Molassembler} types in your
public interface, and any consumers of \texttt{transmogrifier} must also link
against \textsc{Molassembler}, so alter this as needed.

Provided you have installed \textsc{Molassembler} somewhere, this should take
care of include directories, compile definitions and linking for you. Perhaps
you need to specify the path you have installed it to when invoking
\texttt{CMake} using either of the arguments \texttt{CMAKE\_PREFIX\_PATH} or
\texttt{molassembler\_DIR}.

\newpage
\chapter{Using the Python module}

\textsc{Molassembler} provides Python bindings such that the core functionality
of \textsc{Molassembler} can be accessed also via the Python programming
language. In order to build the Python bindings while compiling from source, you
needed to specify \texttt{-DSCINE\_BUILD\_PYTHON\_BINDINGS=ON} when running
\texttt{cmake} during the installation (see also
chapter~\nameref{ch:installation}).

Depending on your method of installation, you may need to specify the path to
the Python module in the environment variable \texttt{PYTHONPATH},
\textit{e.g.}, you might need to run the command 

\begin{Verbatim} 
export PYTHONPATH=$PYTHONPATH:<install directory>/lib/python<version>/site-packages 
\end{Verbatim} 

where \texttt{<version>} is the Python version you are using (\textit{e.g.},
3.6) and \texttt{<install directory>} is where you installed
\textsc{Molassembler}. Now you can simply import the module and use it just
like any other Python module. All functionality available in the Python module
is
\href{https://scine.ethz.ch/static/download/documentation/molassembler/v1.0.0/py}{documented with
examples}\footnote{\texttt{scine.ethz.ch/download/molassembler}}. You can
interactively explore the available functionality in an interactive Python shell
using the \texttt{help} and \texttt{dir} Python functions:

\begin{Verbatim}
import scine_molassembler as masm
dir(masm)
help(masm)
\end{Verbatim}

\chapter{Extensions planned in future releases}
\begin{itemize}
  \item Better handling of aromatics
  \item Better ranking algorithm
  \item Stabilized SMILES parser
\end{itemize}


\chapter{Important references}

Please consult the following references for more details on
\textsc{Molassembler}. We kindly ask you to cite the following reference in any
publication of results obtained with \textsc{Molassembler}.

\vspace{1cm}

% J.-G.~Sobez, M.~Reiher. qcscine/molassembler: Release 1.0.0 (Version 1.0.0), Zenodo, 2020. DOI

J.-G.~Sobez, M.~Reiher. Molassembler: Molecular graph construction, modification
and conformer generation for inorganic and organic molecules, \textit{J. Chem. Inf.
Model.}, 60, 3884 (2020) \href{https://doi.org/10.1021/acs.jcim.0c00503}{DOI}.



%%
% The back matter contains appendices, bibliographies, indices, glossaries, etc.

\backmatter

\bibliography{references}
\bibliographystyle{achemso}

%\printindex

\end{document}
